\part{High Level View of Policies}
\begin{frame}
			 \partpage
\end{frame}

%\section[Resource and Data Governance Policies]{Resource and Data Governance Policies}

%% \begin{frame}
%% 	\frametitle{Overview}
%% 	\begin{itemize}
%% 		\item Login node usage policy
%% 		\item Scratch filesystem purge policy
%% 	\end{itemize}
%% \end{frame}

\section[UH Policies]{UH Policies}
\begin{frame}
  \frametitle{UH Policies}
	\begin{block}{Chapter 708, Hawaii Revised Statutes}
	\begin{itemize}
	\item Access only by authorized people
	\item Should not be used in the act of committing a crime
	\end{itemize}
	\end{block}
	\begin{block}{University of Hawaii Executive Policy E2.210} 
	\begin{itemize}
	\item Protect your password (we will never ask or require your password)
	\item Computer resources should not be used to test or compromise systems without prior authorization
	\item University resources are intended to be used for institutional purposes and may not be used for private gain
	\end{itemize}
	\end{block}
\end{frame}

\section[Mana Usage Policies]{Mana Usage Policies}

\subsection{Login nodes \& DTN nodes}
\begin{frame}
\frametitle{Login nodes \& DTN nodes}
\begin{block}{}
The following actions are acceptable on the shared systems in {\mana}:
\begin{enumerate}
\item File/Directory management [ Login \& DTN nodes ]
\item Text editing with a text editor: vi/vim, emacs, nano [ Login \& DTN nodes ]
\item Transferring files to and from the cluster (scp, rsync, SFTP, globus, aspera, lftp, etc.) [ Login \& DTN nodes ]
\item Shall be used to submit batch and interactive jobs [Login nodes]
\item SSH shell access [ Login \& DTN nodes ]
\end{enumerate}
\end{block}

\begin{block}{}
All other action shall take place on a compute node using an interactive session.  
If any actions outside of the sanctioned activities are detected, the following escalation will take place:
\begin{enumerate}
\item The process will be killed without notice
\item If the user continues this action, we will notify the user of violating the policy
\item If the user continues to ignore our warnings, the user may be banned from the cluster for a duration of time
\end{enumerate}
\end{block}
\end{frame}

\subsection{Storage Policy}
\begin{frame}
\frametitle{Storage Policy}
\begin{enumerate}
\item ITS is not responsible for any data that is deleted or loss due to user error, hardware failure, administrator error.  Users are responsible for their own data and are highly encouraged to backup data that is important to them at other storage locations
\item Files located on the scratch file systems are subject to a \textbf{20 day} purge policy that is based on the file/directory modified timestamp.  \textbf{Files that are purge cannot be recovered}.  Users are encouraged to copy their results off the scratch file system as soon as possible upon completion of their job
\item  Home storage currently snapshots once a day per user in case of accidental file deletion
\end{enumerate}
\end{frame}
%\btVFill
%\begin{center}
%\footnotesize \textbf{\emph{Login node usage policy is subject to change~\\Users will be notified via email prior to changes taking effect}}
%\end{center}


\subsection{User Account Life Cycle}
\begin{frame}
\frametitle{User Account Life Cycle}
\begin{enumerate}
\item Accounts that are idle for 120 days will be locked
\item Accounts that have been locked can be unlocked upon request with no further action then emailing us a request to unlock their account
\item Users that will be leaving UH or are no longer active at UH, will need to have an active UH faculty/staff member in good standing, request a {\mana} account extension.
\end{enumerate}
\end{frame}


%\begin{frame}
%\frametitle{Where to find all Mana specific policies}
%\begin{enumerate}
%%\item \href{http://www.hawaii.edu/infotech/policies/itpolicy.html\#appendixa}{Chapter 708, Hawaii Revised Statutes}
%%\item \href{http://www.hawaii.edu/infotech/policies/itpolicy.html}{University of Hawaii Executive Policy E2.210}
%\item \href{http://go.hawaii.edu/GSY}{Common systems \& Storage}
%\item \href{http://go.hawaii.edu/0SG}{User Account life cycle}
%\item \href{http://go.hawaii.edu/WSG}{Security}
%\item \href{http://go.hawaii.edu/YKG}{For lease storage}
%\item \href{http://go.hawaii.edu/GK0}{Condo Program \& Off Warranty condo nodes}
%%% \item \href{https://docs.google.com/document/d/1mGcCAsmzGT_hWXLcUNfT2LUwiggCnhqrln32k0IGZZI/}{Special allocations}
%\end{enumerate}
%\end{frame}
%
%\begin{frame}
%\frametitle{Login Node Usage Policy}\footnotesize
%The login nodes are a shared resource and are the only access to the cluster for hundreds of user and are meant to provide the following functionality for all users: 
%\begin{itemize}
%\item Providing ssh shell access 
%\item Facilitate the transfer files to and from the cluster:\\Globus, sftp, scp, rsync
%\item Launching and monitoring SLURM jobs (batch and interactive)
%\item Modifying text files with a text editor: vi/vim, emacs, nano
%\end{itemize}
%\bigskip
%\begin{itemize}
%\item[--] If we identify or are notified that a user is running computation on the login nodes, the application will be killed
%\item[--] If we determine that a user repeatedly violates the login node policy, even after being warned, the repeat offender can have their cluster account disabled
%\end{itemize}
%\btVFill
%\begin{center}
%\footnotesize \textbf{\emph{Login node usage policy is subject to change~\\Users will be notified via email prior to changes taking effect}}
%\end{center}
%\end{frame}
%
%
%\subsection{Lustre Filesystem Purge Policies}
%\begin{frame}
%\frametitle{{\lustre} Filesystem Purge Policies}
%Due to users not having any quotas on the {\lustre} filesystem, we need some policy in place which removes older files from the system.  To accomplish this, we have implemented a purge policy.
%
%\begin{itemize}
%\item What in \ctilde/lus/ is subject to the purge policy?
   %\begin{itemize}
   %\item \textbf{Answer:} Everything but directories.
   %\end{itemize}
%\item When will my files be purged?
  %\begin{itemize}  
    %\item \textbf{Answer:} Files greater than or equal to 1 Megabyte in size or are empty (0 bytes) that are 35 days old
    %\item \textbf{Answer:} Files between 1 byte and 1 Megabyte in size that are 120 days old
  %\end{itemize}
%\item What is the age based on?
   %\begin{itemize}
   %\item \textbf{Answer:} Age of file is based on creation time
   %\end{itemize}
   %
%%\item How frequently are files for purged?
%%   \begin{itemize}
%%   \item \textbf{Answer:} Taking into account the fill rate of the {\lustre} filesystem, a purge is performend as usage approaches 80\%.
%%   \end{itemize}
%\end{itemize}
%\btVFill
%\begin{center}
%\footnotesize \textbf{\emph{Purge policy is subject to change~\\Users will be notified via email prior to changes taking effect}}
%\end{center}
%\end{frame}
%
%
%
%\begin{frame}
  %\frametitle{How to check file age for purge}
  %A tool is available on the cluster called age\_check.  This tool will list all files that are older than the defined date you set.
%\begin{semiverbatim}\tiny \texttt
  %$[$root@login-0001 \ctilde$]$\$ age\_check -h
  %
  %Usage:
  %age\_check [-s] [-z] [-d] [-f] [-a <age cutoff>] [-p <path]
%
%
  %Options:
%
  %-s -- Sort output
  %
  %-d -- display creation date per file
  %
  %-z -- display size in bytes per file
  %
  %-f -- Do not exclude files between 1 byte and 1 MB
  %
  %-a age -- Lists files that were created at least age days ago (default is 20 days)
  %
  %-p path -- Directory to check in (default is /lus/scratch/root)
%
  %
  %Example:
  %age\_check -a 30  
%
%\end{semiverbatim}
%
%\end{frame}
   
