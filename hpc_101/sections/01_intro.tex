\part{Introduction}
\begin{frame}
			 \partpage
\end{frame}

\section[University of Hawai'i High Performance Compute Cluster]{University of Hawai'i High Performance Compute Cluster}
\begin{frame}
    \frametitle{University of {\hawaii} High Performance Compute Cluster}
    \begin{itemize}
    \item The University~of~{\hawaii} High Performance Compute Cluster - \textbf{\mana} is \textbf{free} to use for all active faculty, staff, and students affiliated with the University of {\hawaii}
    \item Community acquired nodes are equally accessible to all users
    \item Nodes purchased through the {\textbf{condo program}} are shared with the community, but priority is given to the node owner and their agents
    \itmem Mana is currently undergoing a migration, resource summary subject to change. 
		\item Nodes may be {\textbf{leased}} from the community pool
		\item Additional permanent storage can be leased by faculty \& staff
    \end{itemize}
		\begin{block}{Mana Resource Summary\footnotemark}
  \begin{table}
    \centering
    \resizebox{\textwidth}{!}{%
      \begin{tabular}{l||l||l||l||l||l||l||l}
      %  \multicolumn{8}{c}{ {\large {\textbf{UH-HPC Resource Summary} } } } \\ 
			\toprule                                                                    

              {}               & \thead{\textbf{Nodes}} & \thead{\textbf{CPU Cores}} & \thead{\textbf{Memory}} & \thead{\textbf{GPUs}} & \thead{\textbf{Home/Group Space}} & \thead{\textbf{Scratch Space}} & \thead{\textbf{Storage for lease}} \\ 
							\midrule  \midrule %\hline
        \thead{\textbf{Total}} & 366                    & 9,396                      & 67.3 TB                   &  120                   & 80 TB                             &  111 TB                        & 1 PB              \\ %\hline
				\bottomrule
      \end{tabular}%
    }
  \end{table}
	\end{block}
 \footnotetext{Mana is currently undergoing a migration, resource summary subject to change over the next few months.}
\end{frame}



\subsection{Storage}

%%%%%%%%%%%%%%%%%%%%%%%%%%%%%%%%%%%%%%%%%%%%%%%%%%%%%%%%%%%%%%%%%%%%%%%%%%%%%%%%%%%%%%%%%%%%%%%

%\begin{frame}
  %\frametitle{Storage}
  %{\mana} has two classes and three types of storage that users can potentially access.\\Each type of storage has their own attributes and restrictions
	%\begin{columns}
		%\begin{column}{0.46\textwidth}
		%\begin{block}{Free Storage}
			%\begin{enumerate}
			%\item Permanent Storage
				%\begin{itemize}
				%\item Home Storage
				%\item Group/Lab Storage
				%\end{itemize}
			%\item Scratch Storage
				%\begin{itemize}
				%\item {\lustre}
				%\item Network File System (NFS)
				%\end{itemize}
		%\end{enumerate}
		%\end{block}
		%\end{column}
		%\begin{column}{0.46\textwidth}
		%\begin{block}{For Fee Storage}
			%\begin{enumerate}
				%\item Long Term Storage (LTS)
		%\end{enumerate}
		%\end{block}
		%\end{column}
	%\end{columns}      
%\end{frame}

%%%%%%%%%%%%%%%%%%%%%%%%%%%%%%%%%%%%%%%%%%%%%%%%%%%%%%%%%%%%%%%%%%%%%%%%%%%%%%%%%%%%%%%%%%%%%%%

\begin{frame}
  \frametitle{Permanent Storage}
  \begin{enumerate}
  \item Home Storage
    \begin{itemize}
    \item {\textbf{Purpose}}: Personal storage for applications and active data that needs to persist on {\mana}
    \end{itemize}
  \item Group/Lab Storage
    \begin{itemize}
    \item {\textbf{Purpose}}: Group/Lab storage allows users to share data \& applications with a need for persistence on {\mana}
    \item PI must contact \textbf{uh-hpc-help@lists.hawaii.edu} to set up group/lab storage
    \end{itemize}
  \end{enumerate}
~\\
  Permanent storage options on {\mana} have the following attributes:
  \begin{itemize}
  \item 50 GB default quota with a max quota of 300 GB
  \item Quota increases are re-evaluated annually
  \item Available on all nodes
  \item Freely available to users
  \end{itemize}

\end{frame}


\begin{frame}
  \frametitle{Scratch Storage}
  \begin{enumerate}
    \item {\lustre}
      \begin{itemize}
			\item A parallel distributed file system 
      \end{itemize}
    \item NFS
      \begin{itemize}
			\item A distributed file system 
      \end{itemize}
  \end{enumerate}
~\\
  Scratch file systems on {\mana} have the following attributes:
  \begin{itemize}
	\item 5 TB quota
  \item \textbf{Purge policy} -- 20 days based on file modify time
  \item Available on all nodes
  \item Freely available to users
  \end{itemize}

\end{frame}

\begin{frame}
  \frametitle{Permanent Storage }
  \begin{enumerate}
  \item Long Term Storage (LTS)
    \begin{itemize}
    \item {\textbf{Purpose}}: \footnotesize Additional Personal and/or group storage for applications and active data when all other storage options have been exhausted in {\mana}\footnote{\label{LTS}\tiny\ \url{https://hawaii.kualibuild.com/app/builder/app/6062a8170d0d97001cb7d3af/run}}
    
    \end{itemize}
  \end{enumerate}
~\\
  LTS storage on {\mana} has the following attributes:
  \begin{itemize}
  \item \$20/TB with a minimum purchase of 2TB 
  \item LTS purchases are done in a per year basis
  \item Available on all nodes
  \end{itemize}

\end{frame}

\begin{frame}
  \frametitle{Mana Migration}
~\\
  The new Mana/Koa cluster will have the following attributes:
  \begin{itemize}
	  \item {\textbf{Storage}}: New separate storage system. Files will need to be migrated, and files remaining in Mana will be deleted once the old cluster is turned off. 
	  \item{\textbf{Home}}:Permanent storage with 50GB quota for personal applications and data
	  \item{\textbf{Scratch}}: 800 TB and 400,000,000 file quota shared across all users with a 90 day purge policy. 
  \item{\textbf{Workspace}}: TBD
  \item {\textbf{Timeline}}: Progressive migration in 6 differnt stages \footnotemark
  \end{itemize}
  \footnotetext{\tiny\url{https://datascience.hawaii.edu/mana-upgrade/}}

\end{frame}

%\begin{frame}
  %\frametitle{For Fee Storage}
  %\begin{enumerate}
    %\item ValueStorage
      %\begin{itemize}
      %\item Enterprise class equipment 
      %\item 10 Gbit connected to the UH-HPC
      %\item Currently only accessible on the DTNs and login nodes
      %\end{itemize}
    %\item LTS  
      %\begin{itemize}
      %\item Cheap-n-Deep
      %\item 100 Gbit connected to the UH-HPC
      %\item Accessible on all nodes
      %\end{itemize}
  %\end{enumerate}
%\end{frame}

%%%%%%%%%%%%%%%%%%%%%%%%%%%%%%%%%%%%%%%%%%%%%%%%%%%%%%%%%%%%%%%%%%%%

%\subsection{Networking}
%\begin{frame}
  %\frametitle{Networking}
	%\begin{columns}
	%\begin{column}{0.46\textwidth}
		%\begin{block}{Internal Networks}
		%\begin{itemize}
			%\item Quad Data Rate (QDR)\\InfiniBand{\regtrademark} (IB)   
				%\begin{itemize}
					%\item 40 Gbit
					%\item low latency ($\approx1.3\mu$s)
					%\item non-blocking
				%\end{itemize}
			%\item High Data Rate (HDR) IB   
				%\begin{itemize}
					%\item 100 Gbit
					%\item low latency ($\approx0.5\mu$s)
					%%\item non-blocking
				%\end{itemize}
			%\item 1/25/100 Gbit Ethernet
%
	        %\end{itemize}
                %\end{block}
	%\end{column}
	%\begin{column}{0.46\textwidth}
		%\begin{block}{External Networks}
		%\begin{itemize}
			%\item 100 Gbit SciDMZ
				%\begin{itemize}
					%\item Login \& compute nodes\\connected via a firewall
					%\item DTNs are directly connected
				%\end{itemize}
		%\end{itemize}
                %\end{block}
	%\end{column}
	%\end{columns}
%\end{frame}

%%%%%%%%%%%%%%%%%%%%%%%%%%%%%%%%%%%%%%%%%%%%%%%%%%%%%%%%%%%%%%%%%%%%


\subsection{User Support}
\begin{frame}
  \frametitle{User Support}

  \begin{block}{Online documents \& FAQ}
    Users are encouraged to look through the online documentation \& User Guide prior to contacting ITS-CI directly.  Many questions we receive are repeat questions and we try to capture them in our User Guide.~\\
		\begin{itemize}
		\item \href{http://datascience.hawaii.edu/hpc/}{Information \& policies}
		\item \href{http://go.hawaii.edu/ex2}{Resource Documentation \& User Guide}
		\item \href{http://go.hawaii.edu/3A8}{Video Tutorials}
		\end{itemize}
  \end{block}
  \begin{block}{Contact Information}
    If your question is not answered in our online documentation, please contact us at: UH-HPC-Help@lists.hawaii.edu
    
    \begin{itemize}
    \item For batch jobs \ldots
      \begin{itemize}
      \item[--] Job ID, path to submission script, submission command, error file location, output files
      \end{itemize} 
    \item For other problems \ldots
      \begin{itemize}
      \item[--] State the problem, command issued, host, directory, remote host, error messages
      \end{itemize}
    \end{itemize}
  \end{block}
\end{frame}

\subsection{Condo Program \& Leasing}
\begin{frame}
  \frametitle{What is the Condo Program \& leasing?}
	\begin{block}{Condo Program}\scriptsize
		\begin{itemize}
		\item The condo program allows faculty \& staff to purchase nodes and have them integrated with {\mana}
		\item Condo nodes can take advantage of networking and storage infrastructure that one may not typically have access to
		\item Condo nodes are managed and maintained by ITS staff
		\item Nodes are purchased with a five year warranty
		\end{itemize}
	\end{block}
	\begin{block}{Leasing}\scriptsize
		\begin{itemize}
		\item Nodes leased from the community can have a contract period from one month, up to one year
		\item Node lessees are provided priority access to their leased hardware
		\item Leases are not considered an equipment purchase
		\end{itemize}
	\end{block}
\end{frame}


