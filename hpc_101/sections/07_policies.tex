\section{Policies}
%% \begin{frame}
%% 	\frametitle{Overview}
%% 	\begin{itemize}
%% 		\item Login node usage policy
%% 		\item Scratch filesystem purge policy
%% 	\end{itemize}
%% \end{frame}

\subsection{Use and Management of IT Resources @ UH }
\begin{frame}
  \frametitle{Use and Management of IT Resources @ UH }
  All usage of the {\craycs} must be in compliance with \href{http://www.hawaii.edu/infotech/policies/itpolicy.html\#appendixa}{Chapter 708, Hawaii Revised Statutes} and is subject to University of Hawaii Executive Policy \href{http://www.hawaii.edu/infotech/policies/itpolicy.html}{E2.210}. 
\end{frame}

\subsection{Login Node Usage Policy}
\begin{frame}
\frametitle{Login Node Usage Policy}\footnotesize
The login nodes are a shared resource and are the only access to the cluster for hundreds of user and are meant to provide the following functionality for all users: 
\begin{itemize}
\item Providing ssh shell access 
\item Facilitate the transfer files to and from the cluster:\\Globus, sftp, scp, rsync
\item Launching and monitoring SLURM jobs (batch and interactive)
\item Modifying text files with a text editor: vi/vim, emacs, nano
\end{itemize}
\bigskip
\begin{itemize}
\item[--] If we identify or are notified that a user is running computation on the login nodes, the application will be killed
\item[--] If we determine that a user repeatedly violates the login node policy, even after being warned, the repeat offender can have their cluster account disabled
\end{itemize}
\btVFill
\begin{center}
\footnotesize \textbf{\emph{Login node usage policy is subject to change~\\Users will be notified via email prior to changes taking effect}}
\end{center}
\end{frame}


\subsection{Lustre Filesystem Purge Policies}
\begin{frame}
\frametitle{{\lustre} Filesystem Purge Policies}
Due to users not having any quotas on the {\lustre} filesystem, we need some policy in place which removes older files from the system.  To accomplish this, we have implemented a purge policy.

\begin{itemize}
\item What in \ctilde/lus/ is subject to the purge policy?
   \begin{itemize}
   \item \textbf{Answer:} Everything but directories.
   \end{itemize}
\item When will my files be purged?
  \begin{itemize}  
    \item \textbf{Answer:} Files greater than or equal to 1 Megabytein size  or are empty (0 bytes) that are 35 days old
    \item \textbf{Answer:} Files between 1 byte and 1 Megabyte in size that are 120 days old
  \end{itemize}
\item What is the age based on?
   \begin{itemize}
   \item \textbf{Answer:} Age of file is based on creation time
   \end{itemize}
   
%\item How frequently are files for purged?
%   \begin{itemize}
%   \item \textbf{Answer:} Taking into account the fill rate of the {\lustre} filesystem, a purge is performend as usage approaches 80\%.
%   \end{itemize}
\end{itemize}
\btVFill
\begin{center}
\footnotesize \textbf{\emph{Purge policy is subject to change~\\Users will be notified via email prior to changes taking effect}}
\end{center}
\end{frame}



\begin{frame}
  \frametitle{How to check file age for purge}
  A tool is available on the cluster called age\_check.  This tool will list all files that are older than the defined date you set.
\begin{semiverbatim}\tiny \texttt
  $[$root@login-0001 \ctilde$]$\$ age\_check -h
  
  Usage:
  age\_check [-s] [-z] [-d] [-f] [-a <age cutoff>] [-p <path]


  Options:

  -s -- Sort output
  
  -d -- display creation date per file
  
  -z -- display size in bytes per file
  
  -f -- Do not exclude files between 1 byte and 1 MB
  
  -a age -- Lists files that were created at least age days ago (default is 20 days)
  
  -p path -- Directory to check in (default is /lus/scratch/root)

  
  Example:
  age\_check -a 30  

\end{semiverbatim}

\end{frame}
   
