\section*{Frequently Asked Questions}
\begin{frame}
\frametitle{FAQ}
\begin{itemize}\footnotesize
\item Are files on the cluster backed up?
  \begin{itemize}\tiny
  \item \textbf{Answer:} \textbf{NO!}  User files on the cluster \textbf{\emph{are not backed up}}.  It is up to you as the user to validate and maintain your own backups.  The {\citeam} takes no responsibility for any data that is lost due to human or mechanical error.  
  \end{itemize}

\item Help I have been trying to compile an application without success, can the {\citeam} help?
  \begin{itemize}\tiny
  \item \textbf{Answer:} We are willing to help, just send us an email at \textit{uh-hpc-help@lists.hawaii.edu} with a description of what you have attempted as well as a link to the software you are trying to compile, and/or where it is located on the cluster.
  \end{itemize}
  
\item HELP! My job needs more time and does not support check-pointing.  What can I do?
  \begin{itemize}\tiny
  \item \textbf{Answer:} We understand that jobs may not all fit within the 3 day timelimit or unforeseen issues during the execution can cause a job to run long.  The {\citeam} does review and potentially approves requests to extend jobs in a one off fashion.  A request to extend a job can be denied, since we need to take into account factor such as: how busy the cluster is, the number of time extension requests a user has made in recent history.\\If it is known before the job is submitted that more than 3 days is required, we would ask that you contact us first so we can verify that you truly will need more time.\\If a job is already running and just doesn't look like it will complete in time, please send us a request with the jobid number and what you want the job extended to e.g., please extend my jobs run time to 7 days.\\Please remember the {\citeam} is small and we are also human.  Depending on when you make your request we may not be able to immediately respond or act.  It is best to make your request at the first signs you need more time, in hopes that you can provide us with ample time to act.
  \end{itemize}

\end{itemize}

\end{frame}

\begin{frame}	
\frametitle{FAQ}
\begin{itemize}
  \item Can I run a display program on the cluster?
    \begin{itemize}
      \item \textbf{Answer:} Yes, but not from the login nodes.  To follow cluster usage policies, please run X11 applications from a compute node.  Please see the following steps.
\end{itemize}
\end{itemize}
 \begin{block}{Interactive session with X11}\tiny  
       \begin{enumerate}
      \item Connect via SSH using the -Y option, X11 forwarding enabled
      \item run srun.x11 to start a session on a node
      \end{enumerate}
      \
      ~\\
      ~\\
      $[$local \ctilde$]$\$ ssh -Y user99@uhhpc1.its.hawaii.edu	~\\
      $[$login \ctilde$]$\$ srun.x11 \ddash{}partition sb.q \ddash{}nodes 1 \ddash{}cpus-per-task 1 \ddash{}tasks-per-node 1 \ddash{}time 0-01:00:00	~\\
      $[$compute-0001 \ctilde$]$\$ xterm	~\\
 \end{block}

\end{frame}
