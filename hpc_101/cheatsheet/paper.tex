\documentclass[11pt,twocolumn]{article}



\usepackage[T1]{fontenc}
\usepackage[utf8]{inputenc}
\usepackage{etoolbox}
\usepackage{lmodern}% http://ctan.org/pkg/lm                                                                                                                                                                                                                                                
\usepackage{hyperref}
\usepackage{amsmath}
\usepackage{textcomp}
\usepackage{anyfontsize}
\usepackage{graphicx} % Allows including images                                                                                                                                                                                                                                             
\usepackage{booktabs} % Allows the use of \toprule, \midrule and \bottomrule in tables                                                                                                                                                                                                      
\usepackage{calc}
\usepackage{xstring}
\usepackage{titling}



\newcommand{\lft}{$<$}
\newcommand{\rht}{$>$}

\newcommand{\pth}[1]{{\lft}#1{\rht}}


\newlength{\okinalen}
\setlength{\okinalen}{\widthof{'}}
\newcommand{\okina}{\hbox to.666\okinalen{\hss`\hss}}
\newcommand{\ctilde}{{\fontfamily{ptm}\selectfont\texttildelow}}
\newcommand{\ddash}{-{}-}
\newcommand{\hawaii}{Hawai{\okina}i}
\newcommand{\ci}{Cyberinfrastructure}

\newcommand{\desc}[1]{\small \begin{itemize}\item[]#1\end{itemize}}


\begin{document}
\setlength{\droptitle}{-10em}
\title{HPC--101 -- Cheatsheet \vspace{-2ex}}
\date{\today}
\author{Sean Cleveland Ph.D, Ron Merrill Ph.D,\\David Schanzenbach M.S.\\~\\University of {\hawaii}\\Information Technology Services -- {\ci}}
\maketitle
\section{Directory Navigation}
\begin{itemize}
\item[] \texttt{cd {\pth{Directory}}}\desc{Change to directory specified}
\item[] \texttt{ls {\pth{Path}}}\desc{List the contents of the specified path}
\item[] \texttt{ls -l {\pth{Path}}}\desc{This flag list the paths content in long form}
\item[] \texttt{ls -lh {\pth{Path}}}\desc{Make the filesize humanreable}
\item[] \texttt{ls -a {\pth{Path}}}\desc{Show hidden directories and files}
%\item[] pwd \desc{Print working directory}
\end{itemize}

\section{SSH/SCP}
\begin{itemize}
\item[] \texttt{ssh {\lft}user{\rht}@{\lft}host{\rht}}
\item[] \texttt{ssh -Y {\lft}user{\rht}@{\lft}host{\rht}}\desc{Enable X11 forwarding}
\item[] \texttt{scp~{\lft}file{\rht}~{\lft}user{\rht}@{\lft}host{\rht}:{\lft}remote~path{\rht}}\desc{Copy a file to a remote location}
\item[] \texttt{scp~-r~{\lft}dir{\rht}~{\lft}user{\rht}@{\lft}host{\rht}:{\lft}remote~path{\rht}}\desc{Recursivly copy a directory to a remote location}
\end{itemize}

\section{SLURM}
\subsection{Management And Monitoring}
\begin{itemize}
\item[] \texttt{sbatch \pth{slurm\_script}}\desc{Submit a slurm script to the scheduler for batch processing}
\item[] \texttt{sbatch~--array=[\#-\#]~\pth{slurm\_script}}\desc{Submit a job array to the scheduler}
\item[] \texttt{srun} \desc{Used in submitting an interactive job, or also  used within a slurm\_script}
\item[] \texttt{srun.x11} \desc{Used in submitting an interactive job, that also forwards the X, if you logged in while enabling X forwarding}
\item[] \texttt{squeue} \desc{Display all jobs running or waiting on the cluster}
\item[] \texttt{squeue -u{\lft}user{\rht}} \desc{Only display jobs for the specified user}
\item[]\texttt{sinfo} \desc{Display the partitions and nodes state}
\item[] \texttt{scancel {\lft}jobid{\rht}} \desc{Cancel the specified job}
\end{itemize}

\subsection{Details and Profiling}
\begin{itemize}
\item[] \texttt{scontrol show jobid {\lft}jobid{\rht}}\desc{Show job details of a waiting or running job}
\item[] \texttt{scontrol show parition {\lft}partition{\rht}}\desc{Show details of the specified partition}
\item[] \texttt{scontrol show node {\lft}node name{\rht}}\desc{Show details of the specified node}
\item[] \texttt{sacct -l -j {\lft}jobid{\rht}}\desc{Show usage details of a job}
\item[] \texttt{sacct -u {\lft}user{\rht}}\desc{Show all jobs from a given user}
\end{itemize}

\section{Modules}
\begin{itemize}
\item[] \texttt{module avail} \desc{List all available software modules}
\item[] \texttt{module load \pth{module name}} \desc{Load the specified module into your environment}
\item[] \texttt{module show \pth{module name}} \desc{Shows you what actions the module performs on your environment}
\item[] \texttt{module purge} \desc{unload all loaded modules}
\end{itemize}


\section{Miscellaneous}
\subsection{Commands}
\begin{itemize}
\item[] \texttt{less \pth{file}} \desc{Show contents of file\\Navigate using up/down arrow keys, or spacebar\\Press 'q' to exit out of less}
\item[] \texttt{less -S \pth{file}} \desc{Show contents of file without line wrap\\Navigate using up/down arrow keys, or spacebar\\Press 'q' to exit out of less}
\item[] \texttt{man \pth{command}} \desc{Show the manual entry (if one exists) for the given command\\Uses the same controls as less\\Press 'q' to exit out of man}
\end{itemize}

\subsection{Operators}
\begin{itemize}
\item[] \texttt{|} \desc{Pipe operator is used between two commands, in which you wish to pass the output of the 1st command to the input of the 2nd command\\Ex. \texttt{sacct -l -j 467 | less -S }}
\end{itemize}

\subsection{Hosts}
\begin{itemize}
\item[] login-0001 -- \texttt{uhhpc1.its.hawaii.edu}
\item[] login-0002 -- \texttt{uhhpc2.its.hawaii.edu}
\end{itemize}

\subsection{Globus Endpoints}
\begin{itemize}
\item[] login-0001 -- \texttt{hawaii\#UHHPC1}
\item[] login-0002 -- \texttt{hawaii\#UHHPC2}
\end{itemize}


\end{document}


