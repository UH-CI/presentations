\part{Background}
\begin{frame}
			 \partpage
\end{frame}

\section[What]{What is it?}

\begin{frame}
	\frametitle{What is Open OnDemand \& who is using it?}
%	\vspace{-2.5em}
	\begin{columns}
	\begin{column}{0.50\textwidth}

	\begin{block}{ What is it?}
	Open OnDemand, is a National Science Foundation (NSF) funded science gateway project (OAC1534949, OAC183575) from the Ohio Supercomputer Center, with the goal of making access to HPC resources simpler through a web browser.\footnote[1,frame]{\tiny \href{https://openondemand.org/}{https://openondemand.org/}}
	\end{block}
	\begin{block}{ Who is using it?}
	  \begin{minipage}[t][0.20\textheight][t]{\linewidth} 		
	\begin{itemize}
	\item Universities
	\item The Federal Reserve Bank
	\item Super computing centers
	\end{itemize}
	\end{minipage}
	\end{block}
	\end{column}
	\begin{column}{0.50\textwidth} %%
	\vspace{-2.5em}
	\begin{center}
  \includegraphics[scale=0.15]{05.png}
	\end{center}
	\end{column}
	\end{columns}
\end{frame}


\section[Why]{What can it do for me?}
\begin{frame}
	\frametitle{What can it do for me?}
	\begin{itemize}
	\item Provide a uniform environment which can help simplify instruction in a classroom setting
	\item Built-in file browser with in browser download/upload capabilities
		\begin{itemize}
		\item Max upload file size is currently limited to 5GB
		\end{itemize}
	\item In browser SSH terminal
	\item Job management and monitoring
	\item The ability run applications like Juypter notebook and Rstudio Server on a remote server
	\end{itemize}
\end{frame}

